\section{Introduction}

The demand of crypto assets management become more and more crucial nowadays, especially when manage funds belongs to an organization. It is not good to allow any single person to keep the secret key of the assets in any way. Multisignature, aka Multisig is widely used to address this problem. In Multisig, a single signature should be generated by a group of signing identities, such as a group of board members. Threshold signature technology is used for allowing a subset of identities tolerate to fail or be corrupted. In this technical report, we will investigate the usage of multisignature, threshold signature in blockchain. 

At the end of this report, we layout of our requirements and thinking of crypto assets management for an organization. We proposed a concept of “The better DAO” which aim to leverage the latest crypto technologies to manage Nebulas community fund.

Multisignature is a digital signature scheme which allows a group of users to sign a single document. [wikipedia] In blockchain field, multisignature (often called multisig) is a form of technology used to add additional security for cryptocurrency transactions. Multisignature addresses require another user or users sign a transaction before it can be broadcast onto the blockchain. The required number of signatures is agreed at the start once people agree to create the address. [wikipedia] In blockchain area, multisig add additional security for transactions before it been broadcast onto the blockchain node. The first multisig cryptocurrency wallet was launched in August 2013 by BitGo. [reference] .

Distributed trust is required when manage assets of an organization such as an NGO or a community. The nature of a blockchain system is decentralized and the key idea is the distribution of trust among participants. Distributed trust is not a new idea. In 1979 already argued about the necessity of threshold cryptography for key management. No single entity should be trusted to keep company's secret signature key. However, distributed trust didn’t consider the solution when there is a failure or corruption. In order to tolerate the failure and corruption, threshold signatures is one the major technology is been used. We will discuss threshold signatures in section 2. Later, we will also discuss a few other solutions which leverage the trust in blockchain consensus platform, such as smart contract and chaincode in section 3.

In section 4, we will discuss the requirements and design of the of the fund management system framework. We will introduce the basic requirements of community fund management. And also analyze the failure of the The DAO [reference] in 2016. Afterwards, propose a new concept “The Better DAO” in Nebulas and its implementation based on the requirements.
